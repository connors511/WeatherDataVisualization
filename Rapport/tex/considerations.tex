\chapter{Considerations}
Through the project we had some considerations:
\begin{itemize}
\item Comparing wind farms.
\item List WRK-files in a smart way.
\item File types
\item Radar visualization
\item From and To interval
\item Map design
\item Chart design
\item Windmill icons
\end{itemize}
We were thinking about comparing wind farms, so the client should be able to open more wind farms in different pop ups. The final product only supports one pop up, because it would quickly become confusing.\\
Listing WRK-files in a smart way, like list WRK-files in daily view instead of listing them for every 10th minutes. This wasn't implemented, since we would like the user to have more control, so instead of deleting a file for the whole day, the user can delete files for every 10th minute.\\
We started with the CSV-files for wind farms, since the data was easy to read. After that, we tried WRK-files, which was a little more advanced.\\
We were thinking about having names at the top or bottom of the radar, but it wasn't good when multiple radars very overlapping each other. Now if you play a radar, it appears with a small pop up with radar name and time.\\
We wanted to have two text fields on the map - from and to - so the users could choose an interval. We replaced this with just one text field - From - because we found the second text field inappropriate.\\
Radar buttons for play, pause and reset is also inappropriate, since we have added gestures for it. If you would like to play/pause radar just press on it or to reset the radar, simply just double click on it. This make it a lot easier for the user to control the radar.\\
We had multiple design suggestions for map and chart design. At the beginning we had a scroll-in/out sidebar, which could be show/hidden. This quickly became annoying, so it was replaced by a top bar. It was important that the menu could stay hidden or didn't cover up to much of the map.\\
We wanted to have windmill icons for wind farms on the map. These were very difficult to find on the internet, so we tried to created one by our self in Photoshop. One of the problems were to get a windmill that could be seen on the very bright map. We went back to use an old windmill icon.\\
