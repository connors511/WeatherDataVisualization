% State the problem, your approach and solution, and the main contributions of the paper. Include little if any background and motivation. Be factual but comprehensive. The material in the abstract should not be repeated later word for word in the paper.
Data for weather analysis and forecasts consists of enormous amounts of data.

Comparing these huge amounts of data quickly becomes difficult. Most people interpret images easier than numbers, hence the need for some sort of visualization for the data sets.

The data come from multiple sources (e.g. on site observations from measuring stations, meteorological forecasts from Numerical Weather Prediction models) and, consequently have very diverse formats (time series, georeferenced data, gridded data).

These diverse formats raises and important issue because no common or efficient platform for visualizing and analyzing these data exists except for two earlier attempts: one in MATLAB and one with Google Maps. These earlier application were not flexible enough to be used in a bigger context.

This paper shows the development and considerations throughout the project of an application that might serve as the beginning of a new common open source platform for analyzing, visualizing and comparing weather related data.