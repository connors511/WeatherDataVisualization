\chapter{Body}
\section{The use of frameworks}
Using frameworks because of the decreased development time due to already developed and tested libraries for various tasks such as database accesss, file handling and file uploading.

\section{The C++ application}
\subsection{Why a C++ application?}
We're using a C++ application because it is much faster to parse huge files in C++ than PHP or JavaScript.
The choice of language was because we had some knowledge of both C and C++. C++ is object oriented where as C is a procedural language, and seeing as we all favour object oriented programming, the choice between the two was easy.

We could have chosen other languages such as F, F\#, Java, Perl or Ruby, but our limited knowledge about these made us stick with C++, which is also quite fast and portable.

\subsection{Why Qt?}
We've decided to use Qt because of the support and community around it. It's very mature and most importantly, object oriented.

Qt is a cross-platform application framework developed by Nokia. Popular software such as Skype, VLC media player and VirtualBox along with companies as Google, HP, Philips and Samsung uses it.

Qt makes developing an application with a GUI\footnote{Graphical User Interface} relatively easy, but we've opted to only use the command line part of Qt, because the programming is run by the application and not by a user.

\subsection{Return codes}
The PHP part of the application gets a return code from the C++ application depending on how the parsing went and reacts accordingly. A list of the return codes can be seen in table \ref{tab:cppReturnCodes}.
\begin{table}[htbp]
\centering
\begin{tabular}{|l|l|}
\hline
\textbf{Code} & \textbf{Explanation}\\
\hline
0 & Failure\\
\hline
1 & Success\\
\hline
2 & Success, unkown params ignored\\
\hline
10 & Failed, missing fileid\\
\hline
11 & Failed, missing filename\\
\hline
12 & Failed, missing type\\
\hline
20 & Unknown file type\\
\hline
21 & File not found\\
\hline
22 & Invalid file, or wrongly formatted\\
\hline
30 & Could not read input file\\
\hline
31 & Could not write to input file\\
\hline
32 & Could not read output file\\
\hline
33 & Could not write to output file\\
\hline
34 & Could not delete output file\\
\hline
\end{tabular}
\caption{Return codes from the C++ application}
\label{tab:cppReturnCodes}
\end{table}

\subsection{Extensibility}
The application is coded with extensibility in mind, and all parsers is therefore extending the base class \emph{Parser}.
The \emph{Parser} class contains the basic methods of setting the filename, opening for read and write and closing the open file handle.

\section{The PHP site}
\subsection{Why PHP?}

\subsection{Why FuelPHP?}

\subsection{Speed optimizations}
At the beginning the chart had monthly, weekly and daily view, but when loading all data for one month, the performance were so poor. This was fixed by changing views to 2-weekly, weekly and daily view.
\subsubsection{Caching}
Caching has been implemented both server and client side, to allow for maximum performance when viewing charts and radar image animations.

\begin{table}[htbp]
\centering
\begin{tabular}{|l|l|l|l|}
\hline
\textbf{Area} & \textbf{Before} & \textbf{Server side} & \textbf{Server + client side}\\
\hline
Radar & \textasciitilde 130-150 ms & \textasciitilde 35 ms & \textasciitilde 2 ms\\
\hline
Chart & \textasciitilde 300 ms & \textasciitilde 40 ms & \textasciitilde 2 ms\\
\hline
\end{tabular}
\label{tab:cache_benchmarks}
\caption{Cache benchmarks}
\end{table}

When a radar image or chart data is requested, the server checks if a cache of the output exists. If it does, it checks if the browser already have a copy of the same cache file and send a \textsf{301 Not Modified} response to the browser, telling it to its own file.
If the browser does not have a cache, or it is too old, the server sends the contents of the new cache to the browser.
If no cache is found on the server, or it has expired, the server retrieves the needed info from the database, generate the image or data to output, saves it to the cache and sends it to the browser.


\subsection{The data}
The application support the upload of the following file types:
\begin{description}
\item[csv] Observations form a wind farm. The definition can be found in appendix \ref{ap:csv}.
\item[wrk] Weather image from a radar. The file must obey the new VRIS format. See \cite{VRIS} for the definition of VRIS.
\item[zip] All supported file formats can be zipped to easily upload multiple files at the same time. Unlimited zip and folder nesting is supported\footnote{Note that there might be a limit on the file- or foldername length set by the operating system.}.
\end{description}

\section{Cross-platform compability}
\label{sec:cross-platform}
The use of PHP, C++ and MySQL allows the execution of the application on almost all platforms, because the technologies used are open source and available on most platforms.

We have, however, chosen to drop support for all version of Internet Explorer, because this would increase development time drastically -- time we did not have.