\chapter{Body}
\section{The use of frameworks}
Using frameworks because of the decreased development time due to already developed and tested libraries for various tasks such as database accesss, file handling and file uploading.

\section{The C++ application}
\subsection{Why a C++ application?}
We're using a C++ application because it is much faster to parse huge files in C++ than PHP or JavaScript.
The choice of language was because we had some knowledge of both C and C++. C++ is object oriented where as C is procedural language, and seeing as we all favour object oriented programming, the choice between the two was easy.

We could have chosen other languages such as F, F\#, Java, Perl or Ruby, but our limited knowledge about these made us stick with C++, which is also quite fast and portable.

\subsection{Why Qt?}
We've decided to use Qt because of the support and community around it. It's very mature and most importantly, object oriented.

\subsection{Return codes}
The C++ application returns the following codes to the PHP after processing data:
\begin{table}[htbp]
\centering
\begin{tabular}{|l|l|}
\hline
\textbf{Code} & \textbf{Explanation}\\
\hline
0 & Failure\\
\hline
1 & Success\\
\hline
2 & Success, unkown params ignored\\
\hline
10 & Failed, missing fileid\\
\hline
11 & Failed, missing filename\\
\hline
12 & Failed, missing type\\
\hline
20 & Unknown file type\\
\hline
21 & File not found\\
\hline
22 & Invalid file, or wrongly formatted\\
\hline
30 & Could not read input file\\
\hline
31 & Could not write to input file\\
\hline
32 & Could not read output file\\
\hline
33 & Could not write to output file\\
\hline
34 & Could not delete output file\\
\hline
\end{tabular}
\label{tab:cppReturnCodes}
\caption{Return codes from the C++ application}
\end{table}

\subsection{Extensibility}
The application is coded with extensibility in mind, and all parsers is therefore extending the base class \emph{Parser}.
The \emph{Parser} class contains the basic methods of setting the filename, opening for read and write and closing the open file handle.

\section{The PHP site}
\subsection{Why PHP?}

\subsection{Why FuelPHP?}

\subsection{Speed optimizations}
At the beginning the chart had monthly, weekly and daily view, but when loading all data for one month, the performance were so poor. This was fixed by changing views to 2-weekly, weekly and daily view.
\subsubsection{Caching}

\subsection{The data}
The application support the upload of the following file types:
\begin{description}
\item[csv] Observations form a wind farm. The definition can be found in appendix \ref{ap:csv}.
\item[wrk] Weather image from a radar. The file most obey the new VRIS format. See \cite{VRIS} for the definition of VRIS.
\item[zip] 
\end{description}
