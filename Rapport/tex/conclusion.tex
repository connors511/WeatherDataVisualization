\chapter{Conclusion}
There was created a weather application which met the requirements.
The implementation of different file types and visualizing are done. The database integration, to make it easier and to get higher performance, is done. All this by using open source software.
The implementation of NC-files, was not done due the troubles with CSV and WRK. As earlier stated, this can be implemented.

The implementation of different file types, data visualizing and database integration, to make it easier and to get higher performance, are completed in that way it was required from the beginning. While there is a lot of focus on extensibility and simplicity.
 
The implementation that did not finish, were the NC-files, due the troubles with CSV and WRK. This and coordinates converting were not implemented, due there was no test data with other coordinates, but as earlier stated, this can be implemented.

The use of Git was much better than Dropbox, that was used in the beginning, and allowed all group members to change files without too much concern for conflicts.

The final product ended up being a satisfied solution for a weather application, that could analyse and visualize the enormous amounts of data in an intuitive way.

\begin{table}[htbp]
\centering
\begin{tabular}{| l | l | l | l | l |}
\hline
\multirow{2}{*}{File} & Before & After \\ & & \\ \hline
 & 1 & 2 & 3 & 4 \\ \hline
a & b & c & d & e
\hline
\end{tabular}
\caption{Comparing OpenStreetMap and Google maps}
\label{tab:osm_vs_google}
\end{table}
