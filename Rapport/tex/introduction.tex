\chapter{Introduction}
\label{sec:introduction}

% The Introduction is crucially important. By the time a referee has finished the Introduction, he's probably made an initial decision about whether to accept or reject the paper -- he'll read the rest of the paper looking for evidence to support his decision. A casual reader will continue on if the Introduction captivated him, and will set the paper aside otherwise. Again, the Introduction is crucially important.

% Here is the Stanford InfoLab's patented five-point structure for Introductions. Unless there's a good argument against it, the Introduction should consist of five paragraphs answering the following five questions:

% What is the problem?
The problem with this project is analysing the different kinds of data and visualize it in a smooth and easy way. There are other weather applications on the market, but one of the requirements to create the new weather application, is using open source (OpenStreetMap, Qt etc.). We had different kinds of data in different languages, CSV-files, WRK-files and NC-files. We only did the implementation for CSV-files and WRK-files.

We are very focused on getting high performance and getting this optimized - that was one of the biggest challenges.

% Why is it interesting and important?
This is interesting because there is a lot of data to analyze and to visualize, and to get high performance and user-friendly overview, so this should be created in an intuitive way. There have been combined different programming language to get it optimized. All this and still it should be functional and user-friendly.

% Why is it hard? (E.g., why do naive approaches fail?)
This is tough, because there had to be some changes during its construction, such as implement database integration, to get this optimized. Many of these changes are to satisfy our own requirements through the project.

There are a lot of data to analyze and visualize, so there has be figured out how to do this in the best possible way.

% Why hasn't it been solved before? (Or, what's wrong with previous proposed solutions? How does mine differ?)
There are other applications out there, like DMI\footnote{Danish Meteorological Institute - \url{http://dmi.dk}} and TV2-vejret \footnote{Denmark's nationwide commercial TV channel - \url{http://vejret.tv2.dk}}, but none of these satisfied our requirements that was specified at the beginning. DMI had many different data, but wasn't created in a user-friendly way and TV2-vejret only have weather forecast. There are similar weather programs out there, but none of these have either user-friendly way of handling huge amount data or using open source.

% What are the key components of my approach and results? Also include any specific limitations.
\todo[inline]{What are the key components of my approach and results? Also include any specific limitations?}
This project is going to focus extensibility and simplicity, as key components.

% Then have a final paragraph or subsection: "Summary of Contributions". It should list the major contributions in bullet form, mentioning in which sections they can be found. This material doubles as an outline of the rest of the paper, saving space and eliminating redundancy.