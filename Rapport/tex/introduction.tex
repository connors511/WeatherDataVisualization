\chapter{Introduction}

% The Introduction is crucially important. By the time a referee has finished the Introduction, he's probably made an initial decision about whether to accept or reject the paper -- he'll read the rest of the paper looking for evidence to support his decision. A casual reader will continue on if the Introduction captivated him, and will set the paper aside otherwise. Again, the Introduction is crucially important.

% Here is the Stanford InfoLab's patented five-point structure for Introductions. Unless there's a good argument against it, the Introduction should consist of five paragraphs answering the following five questions:

% What is the problem?
The problem with this project is analysing the different kind of data and visualize it in a smooth and easy way. There are other weather applications on the market, but we had to create a new weather application using opensource (openstreetmap, Qt etc.). We had different kind of data in different language, CSV-files, WRK-files and NC-files. We only did the implementation for CSV-files and WRK-files.\\
We are very focused on getting high performance and getting this optimized - that was one of the biggest challenges.

% Why is it interesting and important?
This is interesting because we have a lot of data to analyse and visualize, and still want high performance and user-friendly overview. We are combining different programming language to get it optimized.

% Why is it hard? (E.g., why do naive approaches fail?)
This is tough, because we had to do some changes during its construction, such as in the beginning we didn't think about database integration. Many of these changes is to satisfy our own requirements through the project.\\
We have a lot of data to analyse and visualize, so we have to figure out how to do this in the best possible way.

% Why hasn't it been solved before? (Or, what's wrong with previous proposed solutions? How does mine differ?)
At the beginning we looked at DMI and TV2-vejret, but none of these satisfied our requirements, that we specified at the beginning. DMI had many different data, but wasn't created in a user-friendly way and TV2-vejret did only have weather forecast. There are similar weather programs out there, but none of these have either user-friendly way of huge amount data or using opensource.

% What are the key components of my approach and results? Also include any specific limitations.


% Then have a final paragraph or subsection: "Summary of Contributions". It should list the major contributions in bullet form, mentioning in which sections they can be found. This material doubles as an outline of the rest of the paper, saving space and eliminating redundancy.