\section{API}
The application has a small API\footnote{Application programming interface} that the application it self uses to retrieve some of the data.

\subsection{Radars}
A list of images for a radar can be retrieved by calling
\begin{lstlisting}[language=sh,caption={Url for retrieving images for a radar}]
GET http://WEBROOT/rest/radar/list.TYPE?lat=LAT&lng=LNG&f=FROM
\end{lstlisting}
Where \textsf{WEBROOT} is the domain name of the server, \text{TYPE} is the return type\footnote{All formats supported by FuelPHPs REST controller is supported. A list of supported formats can be found at \url{http://docs.fuelphp.com/general/controllers/rest.html\#/formats}.\label{fn:fuel_rest}}, \textsf{LAT} is the latitude of the radar, \textsf{LNG} is the longitude of the radar and \text{FROM} is a date from which the data should start from.

The returned data is in array format, where each element is an array consisting of two elements. The first is the timestamp of the image, and the second is the url of the radar image relative to the \textsf{WEBROOT}.

\subsection{Wind farms}
A list of data from a given wind farm can be retrieved by calling
\begin{lstlisting}[language=sh,caption={Url for retrieving data for a wind farm}]
GET http://WEBROOT/rest/csv/list.TYPE?id=ID&c=COL&f=FROM&t=TO
\end{lstlisting}
Where \textsf{WEBROOT} is the domain name of the server, \textsf{TYPE} is the return type\footref{fn:fuel_rest}, \textsf{ID} is ID of the wind farm, \textsf{COL} is the column\footnote{Currently \textsf{PossiblePower}, \textsf{OutputPower}, \textsf{WindSpeed}, \textsf{RegimePossible} and \textsf{OutputRegime}. Note that the columns are case-sensitive.} to fetch data from, \text{FROM} is a date from which the data should start from and \text{TO} is a date at which the data stops at.

The returned data is in array format, where each element is an array consisting of two elements. The first is the timestamp of the image, and the second is the url of the radar image relative to the \textsf{WEBROOT}.

\subsection{Jobs}
A number of files currently in queue to be parsed can be retrieved by calling
\begin{lstlisting}[language=sh,caption={Url for retrieving the number of files queued for process}]
GET http://WEBROOT/rest/jobs/list.TYPE
\end{lstlisting}
Where \textsf{WEBROOT} is the domain name of the server.

\subsection{Upload}
It is possible to send data to the application, just as if they were being uploaded manually. This allows for remote services to send `real time' data to the application. To send data to the server, the following call should be made:
\begin{lstlisting}[language=sh,caption={cURL call to send file}]
curl -F "path=@FILE" http://WEBROOT/api/upload -H "X-KEY: KEY=" -H "X-TIMEZONE: ZONE" -X POST
\end{lstlisting}
Where \textsf{FILE} is the path to the file on the senders machine, \textsf{WEBROOT} is the domain name of the server, \textsf{KEY} is an API key and \textsf{ZONE} is the time zone given by MySQLs Zoneinfo\footnote{See section \ref{sec:application_requirements} for notes about installation of Zoneinfo}.

An example call, to upload \textsf{ekxr20101218\_2350.wrk} with UTC timestamps to the \textsf{WEBROOT} server with the default admin user woudl look like this:
\begin{lstlisting}[language=sh,caption={Example cURL call to send a file}]
curl -F "path=@ekxr20101218_2350.wrk" http://WEBROOT/api/upload -H "X-KEY: YWqmPGH+dOEvOh6pf83a62lzJ1QQLHRMPHhNIaohB3s=" -H "X-TIMEZONE: UTC" -X POST
\end{lstlisting}