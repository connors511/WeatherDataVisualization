\subsection{Coordinates}
\label{sec:coordinates}
There are several ways to locate positions on a map. We have used \emph{conjugate graticule}\footnote{Latitude/Longitude also known as conjugate graticule}, because the test data we had available used this. We have not implemented for other GPS\footnote{Global Positioning System - see \cite{GPS}} coordinates than latitude and longitude, because of the time pressure.

We know that there are other methods to get coordinates - one of them is \emph{Degrees, Minutes, Seconds}. In a circle there are 360 degrees, so to get a minute, each degree is split up into 60 parts, \nicefrac{1}{60}th of a degree. To get seconds, each minute is split up into 60 parts, \nicefrac{1}{60}th of a minute.

To convert \emph{Degrees, Minutes, Seconds} into latitude and longitude, just simply use the formula:
\begin{equation}
Degrees~ + \left(Minutes~ \cdot \nicefrac{1}{60}\right) + \left(Seconds~ \cdot \nicefrac{1}{60} \cdot \nicefrac{1}{60}\right)
\end{equation}

Converting the other way around should be a bit more difficult, but since openstreetmap uses latitude and longitude, this isn't necessary.

This could be implemented by adding a selection, like there has been done with timezones when uploading files.