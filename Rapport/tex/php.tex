\subsection{PHP}

The main programming language chosen for the backend development of the application is PHP. This language has been chosen for several reasons, namely that we are familiar with it, it is well-documented and that it is open source.

\subsubsection{FuelPHP Framework}
From the FuelPHP website\cite{FuelPHP}:
\begin{quote}
FuelPHP is a simple, flexible, community driven PHP 5.3 web framework based on the best ideas of other frameworks with a fresh start.
\end{quote}

FuelPHP uses a HMVC (hierarchical model-view-controller) pattern and comes with different tools for fast and flexible development of web applications. Past experience with the framework and the fact that it takes advantage of the object orientation of PHP has been the main reason why it was chosen. Notable tools and packages that has been heavily used while developing the application are:

\begin{itemize}
\item Oil - a command line utility that can be used for development, such as code scaffolding and running tasks. 
\item Migration - a task that makes it easy to manage database design via. revisions.
\item Task - classes can be run directly in the command line or set up as cron jobs.
\item SimpleAuth - a package containing an authentication driver
\end{itemize}

\subsubsection{Administration}

An administration panel has been created to provide management of the data without having to think of unauthorized use. 

\subsubsection{Tasks}

When first uploading files in the administration, data was parsed in real time, which could lead to a time out by the server, if too many files were uploaded. To avoid this, a task was created to handle the parsing of the data, that is, running DataParser.exe and inserting the data in the database. As the task is running, it will check whether there are any uploaded files, that has not yet got their data parsed - if there are, these will be parsed. This check will be performed every 5 seconds, until the task is stopped manually.

\subsubsection{Timezones}

ANALYSE\todo[fancyline]{Fix}

As a interactive world map with dynamic data sources have been used, timezones must be dealt with in order to show the proper data across the globe and get a consistent UX.

IMPLEMENTATION\todo[fancyline]{Fix}

When a user uploads a file in the administration, he must pick what timezone the data has been observed in. As the dataparser task handles the data for this file, the timezone will be converted to UTC - possible daylight saving time is considered automatically. All data in the application will therefore be in UTC.

\subsubsection{Extendibility}

The application, especially the models, has been developed with flexibility and extendibility in mind. Thus, backend support for new file types can be done with very little code. A file type table has to be created with proper columns for data. A \textsf{model} represent the respective row in the table as an object. Both things can be created automatically in the command line with Oil. KODESTUMP. The created table will have to be related with the \textsf{files} table on sql level. SE DB RELATIONS. 

\subsubsection{Speed optimizations}
At the beginning the chart had monthly, weekly and daily view, but when loading all data for one month, the performance were so poor. This was fixed by changing views to 2-weekly, weekly and daily view.\todo[fancyline]{Write more?}

\subsubsection{Caching}
Caching has been implemented both server and client side, to allow for maximum performance when viewing charts and radar image animations.

\begin{table}[htbp]
\centering
\begin{tabular}{|l|l|l|l|}
\hline
\textbf{Area} & \textbf{Before} & \textbf{Server side} & \textbf{Server + client side}\\
\hline
Radar & \textasciitilde 130-150 ms & \textasciitilde 35 ms & \textasciitilde 2 ms\\
\hline
Chart & \textasciitilde 300 ms & \textasciitilde 40 ms & \textasciitilde 2 ms\\
\hline
\end{tabular}
\label{tab:cache_benchmarks}
\caption{Cache benchmarks}
\end{table}

When a radar image or chart data is requested, the server checks if a cache of the output exists. If it does, it checks if the browser already have a copy of the same cache file and send a \textsf{301 Not Modified} response to the browser, telling it to its own file.
If the browser does not have a cache, or it is too old, the server sends the contents of the new cache to the browser.
If no cache is found on the server, or it has expired, the server retrieves the needed info from the database, generate the image or data to output, saves it to the cache and sends it to the browser.


\subsubsection{The data}
The application support the upload of the following file types:
\begin{description}
\item[csv] Observations form a wind farm. The definition can be found in appendix \ref{ap:csv}.
\item[wrk] Weather image from a radar. The file must obey the new VRIS format. See \cite{VRIS} for the definition of VRIS.
\item[zip] All supported file formats can be zipped to easily upload multiple files at the same time. Unlimited zip and folder nesting is supported\footnote{Note that there might be a limit on the file- or foldername length set by the operating system.}.
\end{description}